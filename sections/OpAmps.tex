\clearpage
\section{Advanced operational amplifier applications}

\subsection{Categories of ideal operational amplifiers}
\begin{table}[htbp]
	\centering
	%TODO: maybe add images
	\begin{tabularx}{0.9\linewidth}{lXX} \toprule
		& Voltage output & Current output \\ \midrule
		Voltage input & Normal Op-Amp \newline $U_a = A_D U_D$ & Transconductance Op-Amp \newline $I_a = S_D U_D$ \\
		Current input & Transimpedance Op-Amp \newline $U_a = Z I_N$ & Current Op-Amp \newline $I_a = k I_N$ \\ \bottomrule
	\end{tabularx}
\end{table}


\subsection{Control loop block diagram}
%TODO: add block diagram in TikZ

\begin{table}[htbp]
	\centering
	\begin{tabular}{ll}
		Open-loop gain & $A_D$ \\
		Closed-loop gain & $A$ \\
		Feedback loop gain & $g$ \\
		Feedback factor & $k_R$ \\
	\end{tabular}
\end{table}

\begin{align}
	U_D &= k_F U_e - k_R U_a \\
	A &= \frac{U_a}{U_e} = \frac{k_F A_D}{1 + k_R A_D} \cong \frac{k_F}{k_R}
\end{align}

\subsubsection{Non-inverting amplifier}
%TODO: add block diagram and/or schematic
\begin{align}
	k_F &= 1 \\
	k_R &= \frac{R_1}{R_1 + R_N} \\
	A &= \frac{U_a}{U_e} = \frac{A_D}{1 + k_R A_D} \cong \frac{1}{k_R} = 1 + \frac{R_N}{R_1}
\end{align}

\subsubsection{Inverting amplifier}
%TODO: add block diagram and/or schematic
\begin{align}
	k_F &= -\frac{R_N}{R_1 + R_N} \\
	k_R &= \frac{R_1}{R_1 + R_N} \\
	A &= \frac{U_a}{U_e} \cong -\frac{R_N}{R_1}
\end{align}

\newpage
\subsubsection{Differential amplifier}
\begin{multicols}{2}
    %TODO: add block diagram and/or schematic
	\begin{center}
    	\begin{circuitikz}[scale=0.8,transform shape]
	\draw 
		(0,0) node[op amp,yscale=-1] (opamp) {}
		(-4,-1) node[anchor=east] {$V_p$} to[R=$R_3$] (-2,-1) |- (opamp.-)
		(-2,-1) to (-2,-2) to[R=$R_4$] (2,-2) |- (opamp.out)
		(-4,1) node[anchor=east] {$V_n$} to[R=$R_1$] (-2,1) |- (opamp.+)
		(-2,1) to[R=$R_2$] (-2,3) node[anchor=south] {$Vcc$}
		(opamp.out) -- (3,0) node[anchor=west] {$Vout$}
	;		
\end{circuitikz}
	\end{center}
    \vfill\columnbreak
    \begin{align}
    	k_F &= ...\\
    	k_R &= \\
    	A &= 
    \end{align}
\end{multicols}

\subsection{Op-Amp building blocks}
\subsubsection{Differential pair}
%TODO: add images
\begin{align}
	I_{C1} &= I_0 + \Delta I = I_0 \left( 1 + \tanh\frac{U_D}{2 U_T} \right) \\
	I_{C2} &= I_0 - \Delta I = I_0 \left( 1 - \tanh\frac{U_D}{2 U_T} \right)
\end{align}

Small signal analysis, for $R_E >> r_S$
\begin{align}
	\frac{u_a}{u_{e1}} &= \frac{R_C}{2 r_S} \\
	\frac{u_a}{u_{e2}} &= -\frac{R_C}{2 r_S}
\end{align}

\subsubsection{Current mirror}
\begin{multicols}{2}
	\begin{center}
		\begin{circuitikz}[scale=0.8,transform shape]
	\draw 
		(0,0) node[anchor=east] {$U_b$} to[R=$R_V$] (2,0)
		(3,-1) node[npn,xscale=-1] (t1) {}
		(6,-1) node[npn] (t2) {}
		(2,0) -| (t1.C)
		(2,0) -| (t1.B)
		(t1.B) -- (t2.B)
		(7,0) -| (t2.C)
		(t1.E) to[R=$R_1$] (3,-3.5) to (3,-4) node[rground] {}
		(t2.E) to[R=$R_2$] (6,-3.5) to (6,-4) node[rground] {}
	;		
\end{circuitikz}
	\end{center}
	\vfill
	\columnbreak
	The emitter current $I_e$ of $T1$ is
	\begin{align}
		I_e = \frac{U_b - U_{BE1}}{R_V + R_1}
	\end{align}
	The output resistance is
	\begin{align}
		r_a = \left.\frac{u_a}{i_a}\right|_{i_e=0} \cong r_{CE2} \left( 1 + \frac{\beta R_2}{R_1 + R_2 + r_{BE2}} \right)
	\end{align}
\end{multicols}

The current mirror is used as current source and active load in differential pairs to replace the resistors. 
It allows to achieve high gains at low supply voltage.
%TODO: add image

\newpage
\subsubsection{Cascode circuit}
\begin{multicols}{2}
	\begin{center}
		%TODO: add schematic
		\begin{circuitikz}[scale=0.8,transform shape]
	\draw 
		(0,0) node[npn] (t2) {$T_2$}
		(0,-2) node[npn] (t1) {$T_1$}
		(0,3) node[anchor=south] {$U_b$} to[R=$R_C$] (t2.C) (0,1) -- (1.5,1) node[anchor=west] {$U_a$}
		(t2.E) -- (t1.C)
		(t1.E) -- (0,-3) node[rground] {}
		(-2,-2) node[anchor=east] {$U_e$} -- (t1.B)
		(-2,0) node[anchor=east] {$U_0$} -- (t2.B)
	;
\end{circuitikz}
	\end{center}
	\vfill
	\columnbreak
	$T_1$ is in common emitter, $T_2$ in common base.
	\begin{align}
		A &= \frac{u_a}{u_e} = A_E \frac{r_{e,B}}{r_{a,E} + r_{e,B}} A_B \nonumber \\
		 &= -\frac{r_{CE1}}{r_{S1}} \frac{r_{S2}}{r_{CE1}+r_{s2}} \frac{R_C}{r_{s2}} \cong \frac{R_C}{r_{s1}}
	\end{align}
	While the gain is identical, the output resistance is higher and the emitter stage has only gain $-1$ and thus only has a miller capacitance of
	\begin{align}
		C_e = C_M \left( 1 + |A_{E,op}| \right) \cong 2 C_M
	\end{align}
\end{multicols}

%TODO: add current mirror improvements, combination with cascode and comparison

%TODO: s impedance conversion fählt au no
